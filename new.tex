\part{Определение новых команд}
\section{Команды без аргументов}

Начнем с примера. Пусть вы пишете текст, в котором регулярно встречается математический значок $\stackrel{\mathrm{def}}{=}$ (он означает <<равно по определению>>). Генерируется этот значок внутри математической формулы такой последовательностью команд:

\begin{verbatim}
	\stackrel{\mathrm{def}}{=}
\end{verbatim}

Часто писать такой длинный набор команд утомительно. Вот бы в \LaTeX’е была предусмотрена команда, скажем, \verb"\eqdef", генерирующая символ бинарного отношения $\stackrel{\mathrm{def}}{=}$! Правда, такой команды нет, но мы ее можем создать. Для этого следует написать так:

\begin{verbatim}
	\newcommand{\eqdef}{\stackrel{\mathrm{def}}{=}}
\end{verbatim}

\newcommand{\eqdef}{\stackrel{\mathrm{def}}{=}}

После того как \TeX прочтет эту строку, он всюду, встречая команду \verb"\eqdef", будет реагировать точно так же, как если бы он видел текст \verb"\stackrel{\mathrm{def}}{=}". Например, формула $x^2 \eqdef x \cdot x$ теперь получается так:

\begin{verbatim}
	x^2 \eqdef x \cdot x
\end{verbatim}

Новая команда \TeX’а, которую мы определили, называется макросом (еще говорят: макроопределение, макрокоманда, макро). Рассмотрим точные правила для создания макросов средствами \LaTeX’а.

Для создания макросов используется команда \verb"\newcommand". Эта команда имеет два обязательных аргумента. Первый из них -- имя, которое вы придумали для вашего макроса. Имена макросов должны подчиняться тем же правилам, что имена \TeX’овских команд: либо backslash и после него одна не-буква, либо backslash и после него -- последовательность букв. Второй обязательный аргумент команды \verb"\newcommand", называемый <<замещающим текстом>>, сообщает \TeX’у смысл макроса: на этот текст ваш макрос будет замещаться в процессе трансляции (как говорят, макрос будет «разворачиваться»).

При пользовании командой \verb"\newcommand" нельзя в качестве имени макроса выбирать имя уже существующей команды или окружения (если вы попробуете так сделать, \LaTeX выдаст сообщение об ошибке). Во втором аргументе команды \verb"\newcommand" (иными словами, в <<замещающем тексте>>) вместе с каждой открывающей фигурной скобкой должна присутствовать соответствующая ей закрывающая, так что определения наподобие

\begin{verbatim}
	\newcommand{\nachatkursiv}{{\itshape}
	\newcommand{\konchitkursiv}{}}
\end{verbatim}
приведут в лучшем случае к сообщению об ошибке (и желаемого эффекта не дадут). Если вам кажется, что такие ограничения стеснительны,
можете изучить по книге [2], как их обходить; для большинства практических целей возможности создания макроопределений, предоставляемые \LaTeX’ом, вполне достаточны.

К сожалению, некоторые русификации \TeX’а не позволяют использовать в именах макросов русские буквы.

Еще одно ограничение: имя новой команды не должно начинаться на end. Наконец, в замещающем тексте макроопределения нельзя пользоваться командой \verb"\verb" или окружением verbatim.

Если команда \verb"\newcommand" дана внутри группы, то смысл определяемой ею новой команды будет забыт \TeX’ом по выходе из группы. Если новая команда определяется в преамбуле, то, естественно, она будет понятна \TeX’у на протяжении всего документа.

Давайте теперь разберем несколько примеров, обращая внимание на типичные ошибки.

Пример типичной ошибки. Пусть в тексте, который вы набираете, регулярно встречаются фразы наподобие Подмногообразия проективного пространства $\mathbf P^n$ -- основной объект изучения алгебраической геометрии, и пусть для сокращения письма вы написали в преамбуле

\begin{verbatim}
	\newcommand{\Pn}{$\mathbf P^n$}
\end{verbatim}

Теперь можно писать, например, так:

\begin{verbatim}
	... пространства~\Pn{} --- основной объект...
\end{verbatim}

Однако для набора формулы $x \in {\mathbf P^n}$ написать \verb"$x\in\Pn$" не удастся: появится сообщение о том, что символ и команда \verb"\mathbf" преступно употреблены вне математической формулы. Причина проста: \TeX исправно подставляет вместо \verb"\Pn" тот <<замещающий текст>>, который вы ему сообщили во втором аргументе команды \verb"\newcommand". В результате этого при развертывании макроса \verb"\Pn" текст \verb"$x\in\Pn$" превращается в незаконный текст \verb"$x\in $\mathbf P^n$$", в котором математическая формула заканчивается со вторым из знаков доллара, а символ оказывается посреди обычного текста. Чтобы можно было напечатать P n не только изолированно, не надо включать знаки доллара в определение:

\begin{verbatim}
	\newcommand{\Pn}{\mathbf{P}^n}
\end{verbatim}

При этом придется, конечно, ставить знаки доллара вокруг \verb"\Pn" в тех случаях, когда в тексте встречается просто P n , но зато наш макрос можно будет использовать и как составную часть более сложных формул.

Есть, впрочем, и более удачный способ борьбы с этой проблемой: определите \verb"\Pn" как

\begin{verbatim}
	\newcommand{\Pn}{\ensuremath{\mathbf{P}^n}}
\end{verbatim}

(без всяких знаков доллара) -- и вы сможете спокойно пользоваться своей новой командой \verb"\Pn" как в тексте, так и в формулах:

\begin{verbatim}
	Пусть \Pn~--- $n$-мерное проективное пространство, а $X\subset \Pn$~--- неприводимое многообразие...
\end{verbatim}

Команда \verb"\ensuremath" всегда обрабатывает свой аргумент как математическую формулу, независимо от того, в тексте или в формуле вы ее используете.

\section{Команды c аргументами}