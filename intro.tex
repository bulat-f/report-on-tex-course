\chapter{Введение}

С точки зрения их записи в исходном тексте, команды делятся на два типа. Первый тип -- команды, состоящие из знака \verb"\" и одного символа после него, не являющегося буквой. Именно к этому типу относятся команды \verb"\{", \verb"\}", \dots , \verb"\%".

Команды второго типа состоят из \verb"\" и последовательности букв, называемой именем команды (имя может состоять и из одной буквы). Например, команды \verb"\TeX", \verb"\LaTeX" и \verb"\LaTeXe" генерируют эмблемы систем \TeX, \LaTeX, \LaTeXe. В имени команды, а также между \verb"\" и именем, не должно быть пробелов; имя команды нельзя разрывать при переносе на другую строку.

В именах команд прописные и строчные буквы различаются. Например, \verb"\large", \verb"\Large" и \verb"\LARGE" -- это три разные команды.

После команды первого типа (из \verb"\" и не-буквы) пробел в исходном тексте ставится или не ставится в зависимости от того, что вы хотите получить на печати:

\quad
\begin{tabular}{ll}
	В чем разница между \$1 и \$ 1?&\verb"В чем разница между \$1 и \$ 1?"\\
\end{tabular}
\quad

После команды из \verb"\" и букв в исходном тексте обязательно должен стоять либо пробел, либо символ, не являющийся буквой (это необходимо, чтобы \TeX смог определить, где кончается имя команды и начинается дальнейший текст).

С другой стороны, если после команды из \verb"\" и букв в исходном тексте следуют пробелы, то при трансляции они игнорируются. Если необходимо, чтобы \TeX все-таки <<увидел>> пробел после команды в исходном тексте (например, чтобы сгенерированное с помощью команды слово не сливалось с последующим текстом), надо этот пробел специально организовать. Один из возможных способов -- поставить после команды пару из открывающей и закрывающей фигурных скобок \{\} (так что \TeX будет знать, что имя команды кончилось), и уже после них сделать пробел, если нужно. Иногда можно также поставить команду \verb"\" (backslash с пробелом после него), генерирующую пробел.
