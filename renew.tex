\chapter{Переопределение существующих команд}

Переопределить значение уже существующей команды с помощью \verb"\newcommand" невозможно. Иногда, однако, это бывает необходимо. Если мы хотим, чтобы в заголовках глав печаталось именно слово <<Глава>>, а не <<Chapter>>, то необходимо определить по-новому команду \verb"\chaptername". Для такого рода целей используется команда \verb"\renewcommand". Она устроена точно так же, как \verb"\newcommand", с тем отличием, что в качестве ее первого аргумента надо указывать имя уже существующей команды; в этом случае по выполнении команды \verb"\newcommand" значение этой команды изменится: она превратится в сокращенное обозначение для текста, указанного в качестве ее второго аргумента. Например, если написать

\begin{verbatim}
	\renewcommand{\alpha}{Ку-ку}
\end{verbatim}
то команда \verb"\alpha" будет генерировать не то, что обычно (букву $\alpha$ в математической формуле и сообщение об ошибке, если эта команда употреблена вне математической формулы), а текст <<Ку-ку>>.

При переопределнии существующих команд с аргументами применяются те же правила, что и в \verb"\newcommand". Место постановки и значение этого необязательного аргумента, а также правила употребления символов \#1, \#2 и т.\,д. при этом будут такие же, как для команды \verb"\newcommand".